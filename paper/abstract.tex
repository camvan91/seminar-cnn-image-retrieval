\begin{abstract}
	This seminar report focuses on using convolutional neural networks for image retrieval. Firstly, we give a thorough discussion of several state-of-the-art techniques in image retrieval by considering the associated subproblems: image description, descriptor compression, nearest-neighbor search and query expansion. We discuss both the aggregation of local descriptors using clustering and metric learning techniques as well as global descriptors.
	
Subsequently, we briefly introduce the basic concepts of deep convolutional neural networks, focusing on the architecture proposed by Krizhevsky \etal \cite{KrizhevskySutskeverHinton:2012}. We discuss different types of layers commonly used in recent architectures, for example convolutional layers, non-linearity and rectification layers, pooling layers as well as local contrast normalization layers. Finally, we shortly review supervised training techniques based on stochastic gradient descent and regularization techniques such as dropout and weight decay.

Finally, following Babenko \etal \cite{BabenkoSlesarevChigorinLempitsky:2014}, we discuss the use of feature activations in intermediate layers as image representation for image retrieval. After presenting experiments and comparing convolutional neural networks for image retrieval with other state-of-the-art techniques, we conclude by motivating the combined use of deep architectures and hand-crafted image representations for accurate and efficient image retrieval.
\end{abstract}
